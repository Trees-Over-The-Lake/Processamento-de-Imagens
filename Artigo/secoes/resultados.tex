\begin{figure}[h]
    \centering
    \includegraphics[width=0.6\textwidth]{Imagens/matriz-confusao.png}
    \caption{Exemplo de matriz de confusão para o classificador.}
\end{figure}

\newpage
\section{Resultados obtidos nos testes exemplos de erros e acertos dos métodos}

Por conta do menu de opções avançadas, o classificador pode ter diferentes 
configurações que afetarão o desempenho e a taxa de acerto do modelo. Os 
resultados encontrados e demonstrados nesta seção são referentes às configurações 
padrão do classificador: aumento do sharpness das imagens em 40\%, aumento do 
contraste em 120\%, aumento do brilho em 10\%, reamostragem para 32 cores, 
suavização gaussiana desligada e 75\% das imagens sendo usadas para o treinamento.

A acurácia média para o classificador tem valor 0.65, com os valores 
oscilando entre 0.6 e 0.7. A especificidade média tem valor 0.33. 
Como pode ser visto na figura 2, os maiores erros do modelo são originados 
da identificação de imagens pertencentes às classes \emph{BI-RADS} 2 e 4. Isso se 
deve à grande semelhança entre as imagens da classe 3 
com as do grupo 2 e 4, e isso faz o modelo “acreditar” fortemente que imagens 
\emph{BI-RADS} pertencentes às classes 2 e 4 pertencem à classe 3. Apesar da etapa de 
pré-processamento auxiliar significativamente o modelo em diferenciar melhor as 
classes, ainda existe uma certa dificuldade do classificador em fazer essa 
distinção. A fim de melhorar o classificador e aumentar a taxa de acerto, 
seriam necessárias imagens com maior resolução tanto para a etapa de treinamento 
quanto para a etapa de avaliação do modelo.
