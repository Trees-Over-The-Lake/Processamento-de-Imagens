\section{Descrição do problema}

O câncer de mama é uma doença que acomete predominantemente as mulheres 
devido a diversos fatores como a exposição prolongada aos hormônios femininos. 
De acordo com o Instituto Nacional de Câncer (INCA) houveram 66.280 casos anuais 
de câncer de mama entre 2020 e 2022. Por conta disto, é importante o acompanhamento 
desde cedo para evitar possíveis complicações. \emph{Breast Imaging Reporting and Data 
System (BI-RADS)} é um sistema padronizado desenvolvido pela \emph{American College of 
Radiology} a fim de auxiliar radiologistas a analisar exames mamográficos e detectar 
sinais prematuros de câncer de mama. 

No sistema \emph{BI-RADS} existem quatro categorias para a composição da mama: Tecido 
complemente composto por gordura, tecido fibroglandular difuso, tecido denso heterogêneo 
e tecido extremamente denso. Os resultados da radiografia da mama retornam imagens 
diferentes para cada uma das quatro categorias, e essas imagens podem ser utilizadas
para treinar um classificador que realiza a identificação da classe \emph{BI-RADS} de um exame 
a fim de auxiliar radiologistas a terem os resultados do exame mais rapidamente. 
Tendo isto em vista, o presente trabalho tem como objetivo o desenvolvimento de um 
classificador capaz de identificar a qual classe \emph{BI-RADS} uma determinada imagem de 
exame mamográfico pertence por meio de descritores de textura. Para classificar uma 
imagem já pré-processada foi utilizada uma \emph{Support Vector Machine}. Em conjunto com 
as descrições das implementações, serão exibidos e avaliados os resultados encontrados 
na etapa de avaliação do classificador, bem como os seus pontos a serem melhorados.

